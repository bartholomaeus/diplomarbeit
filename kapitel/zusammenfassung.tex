%
%
% Kapitel Zusammenfassung
%
%


\chapter{Zusammenfassung und Ausblick}

Die Produktion von Top-Quark-Paaren in Verbindung mit einem Photon er�ffnet die M�glichkeit, die elektromagnetische Kopplung von isolierten Quarks zu untersuchen. Das Top-Quark stellt dabei ein einzigartiges Testobjekt dar, da es zerf�llt, bevor es hadronisieren kann. Damit wird die Suche in neuen Bereichen nach Physik jenseits des Standardmodells erm�glicht. \\
Die hier vorgestellte Analyse untersucht die Separationsf�higkeit mehrerer Kenngr�ssen des Photon-$E_T$-Spektrums hinsichtlich verschiedener Kopplungsst�rken $d_A^{\gamma}$ des elektrischen Dipolmomentes. Die Untersuchungen von Monte-Carlo-Simulationen und aufgenommenen Daten werden im semimyonischen Zerfallskanal von Top-Paaren mit einem zus�tzlichen Photon im Endzustand am CMS-Experiment durchgef�hrt. \\

F�r verschiedene $d_A^{\gamma}$-Szenarien werden mit dem NLO-Ereignisgenerator WHIZARD Matrixelemente berechnet und jeweils 105.000 $t\overline{t}+\gamma$-Ereignisse generiert. Die $E_T$-Spektren der harten Photonen dieser Ereignisse werden mittels dreier Analyseverfahren untersucht: einer 2-Bin-Analyse, der Analyse des gewichteten Mittelwertes der Verteilung und der Analyse einer Exponentialanpassung an das Spektrum. Alle drei Variablen $E_{Low}/E_{High}$, $\overline{E_T}$ und $\lambda$ weisen auf Generatorniveau eine sehr gute Separationskraft auf. \\
Eine CMS Referenzselektion wird als $t\overline{t}$-Vorselektion implementiert, um Ereignisse im semimyonischen Top-Paar-Zerfallskanal zu erhalten. Darin wird ein hochenergetisches, isoliertes Myon und vier Jets gefordert, von denen einer b-artig ist. Ein Signal-zu-Untergrund-Verh�ltnis von $S/B = 2,24$ wird erreicht bei einer Selektionseffizienz von $\epsilon_{t\overline{t}} = 2,8\%$. \\
Zus�tzlich wird eine Photon-Selektion implementiert, die Anforderungen an die Photonidentifikation werden hierbei von einer CMS Referenz �bernommen. Eine Selektionseffizienz f�r $t\overline{t}+\gamma$-Ereignisse von $\epsilon_{\gamma} = 23\%$ und ein Signal-zu-Untergrund-Verh�ltnis von $S/B = 0,53$ werden erreicht. Aus einem Datensatz von 2011 aufgezeichneten Daten, die einer integrierten Luminosit�t von 4,7\,fb$^{-1}$ entsprechen, werden $N_{sel} = 134$ $t\overline{t}+\gamma$-Ereignisse selektiert. F�r alle Schnittvariablen besteht in der Form der Verteilung eine gute �bereinstimmung zwischen experimentell gemessenen Daten und  Monte-Carlo-Simulationen.\\
Die $E_T$-Spektren der selektierten und rekonstruierten Photon-Kandidaten werden f�r verschiedene $d_A^{\gamma}$-Szenarien mit den gleichen Analyseverfahren wie auf Generatorniveau untersucht. Auch auf Rekonstruktionsniveau haben die untersuchten Variablen eine gute Separationskraft. Das $E_T$-Spektrum der 134 Photonen aus Datenereignissen wird analysiert und die Werte f�r $(E_{Low}/E_{High})_{Data}$, $\overline{E_T}_{Data}$ und $\lambda_{Data}$ mit den jeweiligen ermittelten Werten f�r die Monte-Carlo-$d_A^{\gamma}$-Szenarien verglichen. In allen untersuchten Kenngr��en sind die aus Daten ermittelten Werte mit den Standardmodellwerten von $d_A^{\gamma} = 0$ vereinbar. Es ergeben sich die Werte

\begin{alignat*}{2}
(E_{Low}/E_{High})_{Data} &= 5,09 & &\pm 1,19 \\
\overline{E_T}_{Data} &= 55,36 & &\pm 3,19\,GeV \\
\lambda_{Data} &= -0,0333 & &\pm 0,0037 \ .
\end{alignat*}


Systematische Unsicherheiten durch Generatoreffekte und Normierung der Signal- beziehungsweise Untergrund-Datens�tze werden untersucht. Es ergeben sich f�r das Standardmodell-Szenario $d_A^{\gamma} = 0$ die Werte

\begin{alignat*}{3}
E_{Low}/E_{High} &= 6,421 & &\pm 0,304 (\text{stat.}) & &\pm 0,841 (\text{syst.}) \\
\overline{E_T} &= 53,745 & &\pm 1,331 (\text{stat.}) & &\pm 1,381 (\text{syst.})\,\text{GeV} \\
\lambda &= -0,03473 & &\pm 0,00070 (\text{stat.}) & &\pm 0,00208 (\text{syst.}) \ .
\end{alignat*}


Einer der vielversprechendsten Ansatzpunkte zur Verbesserung dieser Analyse ist die Erh�hung der Statistik in den aufgezeichneten Daten. Dies ist derzeit der gr��te limitierende Faktor im Vergleich der hier untersuchten Variablen. Eine Analyse auf Grundlage der 2012 genommenen Daten w�re hier die erste m�gliche Verbesserung, die aufgezeichnete Datenmenge entspricht der vierfachen integrierten Luminosit�t der hier pr�sentierten Daten und die h�here Schwerpunktsenergie von $\sqrt{s} = 8$\,TeV bedeutet au�erdem einen h�heren Wirkungsquerschnitt des in dieser Analyse untersuchten Prozesses. Mit diesen Daten kann eine viel straffere Selektion implementiert werden, in der beispielsweise alle Photonkandidaten in der N�he eines rekonstruierten Jets abgelehnt werden. \\
Weitere systematische Unsicherheiten m�ssen untersucht werden, so zum Beispiel die Jet-Energie-Aufl�sung und der Einfluss der Pileup-Simulation. \\
Um den Einfluss der systematischen Unsicherheiten durch die Wahl des Monte-Carlo-Generators zu verringern, k�nnte eine datengetriebene Untergrundabsch�tzung implementiert werden. So k�nnten in dieser Analyse alle jetbezogenen Photonschnitte durch Template Fits ersetzt werden. F�r die Isolationsvariablen in der Photonselektion k�nnten Schablonen f�r echte Photonen generiert werden, indem man Elektronkandidaten in $Z \rightarrow ee$ Ereignissen ausw�hlt und die sehr �hnliche elektromagnetische Schauerentwicklung von Elektronen und Photonen ausnutzt. Eine Schablone f�r falsch rekonstruierte Photonen k�nnte aus Jetkandidaten in QCD-multijet Ereignissen erhalten werden, welche die Photonselektionsanforderungen erf�llen. \\
