%
%
% Kapitel Zusammenfassung
%
%


\chapter{Zusammenfassung}





%Alle untersuchten Variablen, der Wirkungsquerschnitt des Prozesses und die Analysen des $E_T$-Spektrums, zeigen eine Korrelation zum $d_A^{\gamma}$-Wert des elektrischen Dipolmomentes und sind damit prinzipiell als diskriminierende Variable geeignet. Es ist zu bedenken, dass bei der Analyse von echten Daten nur eine geringe Anzahl von Ereignissen betrachtet werden und dadurch schon wenige statistische Ausreisser einen gro�en Einfluss auf das Ergebnis der Analyse haben k�nnen. Dies gilt im Besonderen f�r die Analyse des Schwerpunktes der $E_T$-Verteilung.