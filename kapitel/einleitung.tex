%
%
% Kapitel Einleitung
%
%


\chapter{Einleitung}

Ende 2009 begann der Large Hadron Collider (LHC) in Genf mit der Datennahme. Damit beginnt eine �beraus interessante Zeit f�r die Teilchenphysik, in der Bekanntes �berpr�ft und viel Neues entdeckt werden kann. Das erste, mit Spannung erwartete Resultat war die Entdeckung eines Higgs-artigen Bosons im Sommer 2012 \cite{CMS:Higgs}. Dieses vermittelt den Austauschteilchen der schwachen Wechselwirkung, den $W^{\pm}$ und dem $Z$, Masse und wurde schon 1964 theoretisch im Higgs-Formalismus eingef�hrt \cite{Higgs:Bosons}. Das Higgs-Boson ist das letzte fehlende Teilchen im Standardmodell der Teilchenphysik, welches sich seit fast 40 Jahren als sehr zuverl�ssig erwiesen hat \cite{Weinberg:SM}. Das Top-Quark wurde als bisher letztes Teilchen des Standardmodells 1995 durch die Experimente CDF \cite{CDF:Top} und D\O \cite{D0:Top} am Fermilab nachgewiesen. \\
Im Hinblick auf die Untersuchung von Ph�nomenen jenseits des Standardmodells besitzt das Top-Quark einige vielversprechende Eigenschaften. Aufgrund seiner hohen Masse wird erwartet, dass die Effekte solcher Ph�nomene auf die Kopplungen des Top-Quarks gr��er sind als f�r andere Quarks \cite{Aguilar:Couplings}. Eine herausragende Eigenschaft des Top-Quarks ist au�erdem, dass es zerf�llt, bevor es hadronisieren kann, so dass diese Effekte mit der vom LHC bereitgestellten hohen Schwerpunktsenergie und Luminosit�t sichtbar gemacht werden k�nnen. \\
In dieser Arbeit wird die Kopplungsst�rke des elektrischen Dipolmomentes $d_A^{\gamma}$ untersucht. Dazu werden vom CMS-Detektor in 2011 bei einer Schwerpunktsenergie von $\sqrt{s}=7$\,TeV aufgenommene Daten, die einer integrierten Luminosit�t von $\mathcal L_{int} = 4,7$\,fb$^{-1}$ entsprechen, herangezogen. Die hier vorgelegte Diplomarbeit beginnt mit einem kurzen �berblick �ber die theoretischen Grundlagen dieser Analyse. Es folgt eine �bersicht des LHC und eine Beschreibung des CMS Experimentes. Danach wird das Konzept der vorliegenden Analyse dargelegt, deren Durchf�hrung in den nachfolgenden Kapiteln detailliert vorgestellt wird. Im vorletzten Kapitel wird diskutiert, wie gro� die Separationskraft der untersuchten Variablen zwischen verschiedenen $d_A^{\gamma}$-Szenarien ist und abschlie�end im letzten Kapitel eine Zusammenfassung der Arbeit gegeben. Diese vermittelt einen �berblick der wichtigsten Erkenntnisse und einen Ausblick auf m�gliche Verbesserungen. \\

\subsubsection*{Anmerkung}

In dieser Arbeit wird das nat�rliche Einheitensystem verwendet. Damit gilt:
\begin{equation*}
c = \hbar = 1 \ .
\end{equation*}
Die Einheiten einiger h�ufig benutzter Gr��en werden damit zu
\begin{equation*}
[\text{Energie}] = [\text{Impuls}] = [\text{Masse}] = [\text{Zeit}]^{-1} = [\text{L�nge}]^{-1} = \text{eV} \ .
\end{equation*}
Das beim CMS Experiment verwendete Koordinatensystem hat seinen Ursprung im Wechselwirkungspunkt. Dabei zeigt die $x$-Achse zum Zentrum des LHC, die $y$-Achse vertikal nach oben und die $z$-Achse entlang der Strahlachse. Der Polarwinkel $\theta$ nimmt auf der positiven $z$-Achse den Wert Null an, auf der negativen den Wert $\pi$. H�ufig wird er zur Pseudorapidit�t $\eta$ transformiert, die wie in Abschnitt~\ref{sec:detektor} beschrieben definiert ist. Der Azimutalwinkel $\phi$ wird in der $x-y$-Ebene gemessen, welche senkrecht zur Strahlachse steht. Desweiteren wird die Einstein'sche Summenkonvention verwendet.