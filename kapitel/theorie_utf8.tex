%
%
% Kapitel Theorie
%
%


\chapter{Das Top-Quark am LHC}

\section{Das Standardmodell der Teilchenphysik}

Das Standardmodell der Teilchenphysik beschreibt die Elementarteilchen, aus denen die Materie aufgebaut ist, als Fermionen sowie die zwischen ihnen wirkenden Kräfte durch Austauschbosonen. 

\begin{table}%
\caption{Die drei Teilchenfamilien von Quarks und Leptonen mit ihren Quantenzahlen. Q bezeichnet die elektrische Ladung in Einheiten der Elementarladung, $T_3$ die dritte Komponente des schwachen Isospins, $Y$ die Hyperladung \cite{Kuessel:Diplom}.}
\centering
\begin{tabular}{|c||c|c|c|c|c|c|c|}
\hline
 & 1 & 2 & 3 & $Q\,[e]$ & $T_{3}$ & $Y$ & Farbe \\
\hline
\hline
\multirow{2}{*}{Quarks} & \multirow{2}{*}{ $ \left( \begin{array}{cc}
u  \\ d  \\ \end{array} \right)_{L} $} & \multirow{2}{*}{ $ \left( \begin{array}{cc}
c  \\ s  \\ \end{array} \right)_{L} $} & \multirow{2}{*}{ $ \left( \begin{array}{cc}
t  \\ b  \\ \end{array} \right)_{L} $} & 2/3 & 1/2 & 1/3 & \textit{rgb} \\
 & & & & -1/3 & -1/2 & 1/3 & \textit{rgb} \\
 & u$_{R}$ & c$_{R}$ & t$_{R}$ & 2/3 & 0 & 4/3 & \textit{rgb} \\
 & d$_{R}$ & s$_{R}$ & b$_{R}$ & -1/3 & 0 & -2/3 & \textit{rgb} \\
\hline
\multirow{2}{*}{Leptonen} & \multirow{2}{*}{ $ \left( \begin{array}{cc}
\nu_e  \\ e^-  \\ \end{array} \right)_{L} $} & \multirow{2}{*}{ $ \left( \begin{array}{cc}
\nu_{\mu}  \\ \mu^-  \\ \end{array} \right)_{L} $} & \multirow{2}{*}{ $ \left( \begin{array}{cc}
\nu_{\tau}  \\ \tau^-  \\ \end{array} \right)_{L} $} & 0 & 1/2 & -1 & - \\
 & & & & -1 & -1/2 & -1 & - \\
 & $\nu_{e,R}$ & $\nu_{\mu,R}$ & $\nu_{\tau,R}$ & 0 & 0 & 0 & - \\
 & e$_{R}^-$ & $\mu_{R}^-$ & $\tau_{R}^-$ & -1 & 0 & -2 & - \\
\hline
\end{tabular}
\label{tab:fermionen}
\end{table}

\section{Das Top-Quark am LHC}

\section{Anomale Kopplungen des Top-Quarks}